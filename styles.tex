%%% Макет страницы %%%
\geometry{a4paper,top=2cm,bottom=2cm,left=3cm,right=2cm}

%%% Кодировки и шрифты %%%
\renewcommand{\rmdefault}{ftm} % Включаем Times New Roman

%%% Выравнивание и переносы %%%
\sloppy					% Избавляемся от переполнений
\clubpenalty=10000		% Запрещаем разрыв страницы после первой строки абзаца
\widowpenalty=10000		% Запрещаем разрыв страницы после последней строки абзаца

%%% Библиография %%%
\makeatletter
\bibliographystyle{utf8gost705u}	% Оформляем библиографию в соответствии с ГОСТ 7.0.5
%%%\renewcommand{\@biblabel}[1]{#1.}	% Заменяем библиографию с квадратных скобок на точку:
\makeatother

%%% Изображения %%%
\graphicspath{{images/}} % Пути к изображениям

%%% Цвета гиперссылок %%%
\definecolor{linkcolor}{rgb}{0,0,0.9}
\definecolor{citecolor}{rgb}{0.1,0.1,0.9}
\definecolor{urlcolor}{rgb}{0,0,1}
\hypersetup{
    colorlinks, linkcolor={linkcolor},
    citecolor={citecolor}, urlcolor={urlcolor}
}

%таблицы
\pgfkeys{/pgf/number format/.cd,precision=2,use comma,fixed,1000 sep={}}
\pgfplotsset{compat=newest} % использовать новые возможности pgfplots
\usetikzlibrary{positioning,arrows}

\tikzstyle{algorithm} = [rectangle,
                      thick,
                      minimum size=1cm,
                      draw=red!50!black!50,
                      top color=white,
                      bottom color=red!50!black!20,
                      font=\itshape]

%%% Оглавление %%%
\renewcommand{\cftchapdotsep}{\cftdotsep}
\linespread{1.3}


\lstset{ %
 basicstyle=\footnotesize\ttfamily,        % the size of the fonts that are used for the code
  breakatwhitespace=false,         % sets if automatic breaks should only happen at whitespace
  breaklines=true,                 % sets automatic line breaking
  captionpos=b,                    % sets the caption-position to bottom
  frame=none,                    % adds a frame around the code
  keepspaces=true,                 % keeps spaces in text, useful for keeping indentation of code (possibly needs columns=flexible)
  language=Octave,                 % the language of the code
  numbers=left,                    % where to put the line-numbers; possible values are (none, left, right)
  numbersep=15pt,                   % how far the line-numbers are from the code
  stepnumber=1,                    % the step between two line-numbers. If it's 1, each line will be numbered
  tabsize=4,                       % sets default tabsize to 2 spaces
  title=\lstname                   % show the filename of files included with \lstinputlisting; also try caption instead of title
}