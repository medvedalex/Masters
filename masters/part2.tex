\chapter{Алгоритмы генерации и обработки данных} \label{chapt2}

В этой главе описываются постановки задач аппроксимации данных, а также исследованные алгоритмы генерации 
синтетических данных и
экстраполяции тарировочных данных. 


\section{Подходы к аппроксимации функций}\label{sect2_1}
% поправить слова
Метод наименьших квадратов - один из базовых методов для оценки неизвестных 
параметров моделей по набору данных, при этом исследуется на минимум 
следующая функция


\begin{center}
 $ s(\vec{p}) = \left| f(x_t, \vec{p}) - y_t \right| ^ 2 \rightarrow min $
\end{center}


Где $\vec{p}$ --- вектор оцениваемых параметров модели, $f(x_t, \vec{p}) $
--- функция, аппроксимирующая значения $y_i$ в точках $x_i$.

Во многих случаях существует аналитическое решение для системы $M$ уравнений 
$m \in [0..M]$
\begin{center}
 $ \displaystyle\sum_{t = 1}^n \left( y_t - f(x_t,\vec{p})\right) 
 \frac{\partial f(x_t, \vec{p})}{\partial b_m} = 0 $
\end{center}

\textbf{Постановка линейной задачи меньших квадратов}

Если зависимость модели от параметров $\vec{b}$ имеет вид 

\begin{center}
$ y_t = \displaystyle\sum_{j=1}{M}p_j x_tj + \epsilon_t = 
x^T_tb + \epsilon_t$ 
\end{center}
, то такая задача называется линейной. Эта задача решается аналитически, 
её решение можно найти в книгах по статистике, например[].
%Линник Ю. В. Метод наименьших квадратов и основы математико-статистической 
%теории обработки наблюдений. — 2-е изд. — М., 1962. (математическая теория)



\textbf{Нелинейная задача меньших квадратов}


В общем случае, решения системы дифференциальных уравнений нет и 
применяются численные методы решения оптимизационных задач, 
основанные на градиентном спуске.

\textbf{Критерий Зюбина-Петухова}
Известно, что при мягкое излучение поглощается в одном-двух метрах породы [ссылка], мягкое излучение обладает
меньшей проникающей способностью, чем мюоны, но оказывает влияние на измерение(статистическая погрешность),
поэтому измерения небольших глубинах (до нескольких метров) должно меньше учитываться для тарировки. 
Напротив, ошибки на больших глубинах должны учитываться сильнее, поскольку значение интенсивности  меньше в разы
(в e раз на 10 м. в. э.). Такими свойствами обладает критерий Зюбина-Петухова, минимизирующий следующую функцию : 

\begin{center}
$s(\vec{p}) = \displaystyle\sum_{t=0}^N \left|
\frac{\phi(x_t, \vec{p}) - y_t}{min(\phi(x_t, \vec{p}), y_t)}\right|^2 
\frac{100\%}{N} \rightarrow min$ %enum
 
\end{center}


\section{Постановка задачи экстраполяции тарировочных кривых}\label{sect2_2}

Качество обработки результатов измерения существенно влияет на погрешность
получаемых результатов, которая в первую очередь определяется
качеством восстановления зависимости интенсивности потока мюонов от глубины в м. в. э.

Тарировочные данные, полученные при получении тарировочной кривой на воде, 
показывают, что интенсивность потока мюонов падает монотонно с
глубиной, а первая производная интенсивности монотонно возрастает 
от отрицательных значений, к нулю. Этот факт послужил основанием для отказа 
от сплайновой интерполяции, практикуемой ранее. Используя решение 
однородного уравнения переноса, имеющего экспоненциальный характер, было решено
искать целевую функцию в виде суммы экспонент:
\begin{center}
$\mathit{ f(t)  = \displaystyle\sum_{j=0}^M a_j exp(-b_j t) }$ 
\end{center}


% Критерии, кроме того нужно рассказать про non-liner aleast squares

В качестве критерия была выбрана следующая норма (Зюбина--Петухова ): 
\begin{center}

$s(\vec{a}, \vec{b}) = \displaystyle\sum_{i=0}^N \left|
\frac{f(a_j, b_j, i) - y_i}{min(f(a_j, b_j, i), y_i)}\right|^2 
\frac{100\%}{N} \rightarrow min$ %enum

\end{center}
Данный критерий похож на взвешенную задачу наименьших квадратов 
нелинейных функций, в иностранной литературе можно встретить следующее название 
Weightened Non-Linear Least Squares Problem. Эта задача отличается
от задачи минимизации наименьших квадратов делителем зависящим от номера измерения : 
\begin{center}
$w(\vec{a}, \vec{b}) = \displaystyle\sum_{i=0}^N \frac{\left|f(a_j, b_j, i) - y_i\right|^2}{w_i} \rightarrow min$
\end{center}
Один из способов решения задачи - взешивание измерений, затем 
используется любой из существующих методов для нахождения 
наименьших квадратов нелинейных функций.


Однако, поскольку в критерии Зюбина-Петухова в знаменателе 
находится функция зависящая от минимизируемых параметров, 
данный подход не работает. Упростим $k$-е слагаемое выражения %enum
(для краткости опустим постоянные множители $N$, $100\%$) : 

\begin{center}
 \LARGE{$ \left|\frac{f(a_j, b_j, k) - y_k}{min(f(a_j, b_j, k), y_k)}\right|^2 = \left\{ {
    \left| 1 - \frac{y_k}{f(a_j, b_j, k)}\right|^2 , f(a_j, b_j, k) < y_k  \atop 
    \left| 1 - \frac{f(a_j, b_j, k)}{y_k}\right|^2 , f(a_j, b_j, k) > y_k  
 } \right. $}
\end{center}

Таким образом, разбивая $ 2 M $--мерное пространство параметров $ \vec{a}, \vec{b}$ на две области, мы можем сформулировать критерий в терминах задачи наименьших квадратов нелинейных функций.
Рассмотрим кусочно-гладкую функцию $\phi(\vec{a}, \vec{b})$: 
\begin{center}
$ \phi(\vec{a}, \vec{b}) = \left\{ {
    \frac{y_k}{f(a_j, b_j, k)} , f(a_j, b_j, k) < y_k  \atop 
    \frac{f(a_j, b_j, k)}{y_k} , f(a_j, b_j, k) > y_k  
 } \right.$
\end{center}

Решая задачу о наименьших квадратах для функции 
$\phi(\vec{a}, \vec{b})$ на постоянном векторе данных, заполненным
единицами получаем решение для минимизации нормы 
Зюбина-Петухова, используя стандартные методы: 
$v(\vec{a}, \vec{b}) = \displaystyle\sum_{i=0}^N \left|1 -
\phi(\vec{a}, \vec{b})\right|^2 \rightarrow min \rightarrow min$


\section{Генерация синтетических данных}\label{sect2_3}


Получение большого набора тестовых измерений (исходных данных 
для тарировочных кривых) сопряжено с рядом трудностей. Поскольку 
плотномер находится в активной разработке,
большую часть времени прибор недоступен для проведения тестовых
измерений. Кроме того, примерное время одной серии измерений составляет
около часа. Соответственно 
для получения большего набора данных время растёт пропорционально.


Ввиду перечисленных сложностей, было решено в качестве тестовых
данных для алгоритмов использовать синтетические данные. Было 
рассмотрено два подхода моделирования данных -- 
полная симуляция потока мюонов, генерация зашумленных данных 
на основе известной целевой функции (суммы монотонно убывающих экспонент).


\subsection{Программный пакет MUSIC}\label{subsect2_3_1}


Был исследован ряд статей и монографий описывающих подходы и существующие
решения по моделированию мюонов <ссылки>. В результате в качестве ПО для 
генерации потока мюонов
был выбран программный пакет MUSIC (MUon SImulation Code).Он обладает
рядом достоинств -- 
%тут надо поправить текст/найти индекс цитирования
результаты его моделирования находятся в соответствии с экспериментальными
данными в широкой области от нескольких ГэВ до 1 ТэВ (тогда как ряд
моделей <ссылка> обладают
недостатком мюонов в определенных областях энергий), программный 
пакет доступен бесплатно, доступен его исходный код. Автор пакета
Кудрявцев В.А. <ссылка> дал несколько 
советов по моделированию потока мюонов в среде.


Программный пакет MUSIC проводит моделирование в 3х измерениях с помощью 
метода Монте-Карло. Взаимодействие мюонов с материей с высокими
потерями энергии рассматриваются 
как стохастические процессы. При этом учитываются угловое отклонение 
и смещение мюонов из-за множественного рассеяния на ядрах атомов, 
потери энергии на тормозное излучение
и неупругое рассеяние. В данной работе для каждого тестового измерения 
проводилась симуляция 5000 мюонов и из статистики определялась вероятность
выживания мюонов на заданной глубине. 
Для тестирования алгоритмов было проведено 34 серии измерений по 10 измерений в серии. 


\subsubsection{Генерация данных на основе целевой функции}\label{subsect2_3_2}


Проверка алгоритмов на основе симуляции потока мюонов обладает 
одним недостатком -- неизвестна зависимость флуктуаций от приближаемой 
кривой зависимости интенсивности 
потока от глубины. По этой причине была проведена другая серия 
синтетических измерений. В этой серии из допустимого диапазона 
параметров случайно определялись параметры 
экспонент, определялся "поток мюонов" на глубине и затем к этим 
данным добавлялся шум в пределах 5\% относительной погрешности.



\section{Алгоритмы экстраполяции тарировочной кривой}\label{sect2_4}

В ходе работы были рассмотрены следующие алгоритмы:

\begin{itemize}
 \item Прони-подобные алгоритмы
 \subitem Алгоритм Прони
 \subitem Модифицированный алгоритм Прони(Осборн)
 \subitem Алгоритм Кунга
 \item Алгоритм Левенберга-Марквардта
 \item Модификация жадного алгоритма
 
\end{itemize}

Описание алгоритмов дается в соответствующих секциях

\subsection{Алгоритм Прони}\label{subsect2_4_1}
Prony's method) was developed by Gaspard Riche de Prony in 1795. However, practical use of the method 
awaited the digital computer.[1] Similar to the Fourier transform, Prony's method extracts valuable information
from a uniformly sampled signal and builds a series of damped complex exponentials or sinusoids. This allows for 
the estimation of frequency, amplitude, phase and damping components of a signal.

Алгоритм Прони был разработан Гаспаром Рише де Прони в 1795 году. Чаще всего этот метод рассматривается в качестве метода анализа сигналов (выделения экспоненциально--затухающих
синусоидальных гармоник), но так же 
\subsection{Модифицированный алгоритм Прони(Осборн)}\label{subsect2_4_2}

\subsection{Алгоритм Кунга}\label{subsect2_4_3}

\subsection{Алгоритм Левенберга-Марквардта}\label{subsect2_4_4}

\section{Архитектура системы автоматизации обработки данных}\label{subsect2_5}

Рабочее место оператора состоит из <>. Оператор получает файл с тарировочными кривыми, затем АРМ оператора считает рисует график проводится измерение
\clearpage