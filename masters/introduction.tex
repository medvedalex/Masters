\chapter*{Введение}							% Заголовок
\addcontentsline{toc}{chapter}{Введение}	% Добавляем его в оглавление

Один из основных параметров грунта, используемый в инженерной геологии и строительстве, ---
плотность. Информация о плотности грунта 
определяет объемы и состав работ при подготовке к строительству зданий, влияет на заключения о 
безопасности проводимых строительных работ, и качестве их выполнения\cite{gost5180, souzdornii}. 

Плотность грунта измеряется либо контактными методами через замер плотности образцов, 
либо бесконтактными радиационными методами.

Привлекательность первого способа – присущая непосредственным измерениям точность, недостаток ---
локальный характер измерения (характерный объем забора – 1 дм$^{3}$), невозможность оценки объекта в целом,
и вытекающий отсюда чрезмерный объем бурильных работ.

Радиационный метод обеспечивает комплексную оценку исследуемого объекта, используется в широком классе задач ---
спектрометрическом контроле газовых и водных сред, дефектоскопии, рентгено-структурном анализе материалов и пр \cite{gammaquant}. 
В частности, метод подходит для определения объемного веса большинства петрографических типов пород, и практически 
незаменим для измерения плотности дисперсных грунтов. 

Однако использование гамма-плотномеров, реализующих радиационный метод измерения, связано с серьезными требованиями 
к безопасности \cite{gost23061}, вызванными наличием в составе устройства источника радиоактивного излучения, --- необходимостью 
специально оборудованных мест хранения, доставки и множественных согласований с санитарно-эпидемиологическими службами. 
Эти обстоятельства стимулируют исследователей искать альтернативные подходы к измерению плотности грунта. 

Один из недавно предложенных способов, исследуемый в Институте автоматики и электрометрии СО РАН, ---
мюонный скважинный плотномер \cite{patentdensitometer}. За счет конструктивных решений удалось обеспечить высокую чувствительность 
датчика плотномера при измерении потока атмосферных мюонов, что позволяет сочетать приемлемое время измерения, 
безопасность эксплуатации прибора и комплексный характер получаемых оценок для плотности грунта.

На данный момент пользователь мюонного плотномера вынужден строить тарировочные кривые на миллиметровой бумаге 
и наносить данные на графики вручную. Соответственно, экстраполяция проводится «на глаз», что приводит к высокой 
погрешности при обработке данных.

%Для повышения качества обработки данных авторы прибора предложили алгоритм, автоматизирующий аппроксимацию тарировочных 
%кривых вариацией метода градиентного спуска. Используемый алгоритм аппроксимации тарировочных 
%данных обеспечивает нахождение аппроксимирующей кривой с погрешностью восстановления  
%около двух процентов в течение 10 минут. Однако, вопрос автоматизации и снижения погрешности обработки данных по-прежнему стоит на повестке дня.
%\newpage

Цель работы --- разработка системы автоматизации обработки данных для мюонного плотномера. 
Данная цель достигается посредством решения следующих задач:

\begin{itemize}
 \item Анализ специфики измерений мюонным плотномером 
 \item Построение физической модели генерации и поглощения средой атмосферных мюонов
 \item Формирование требований к системе автоматизации обработки данных
 \item Разработка модуля генерации синтетических данных 
 \item Определение алгоритмов обработки данных 
 \item Разработка архитектуры системы автоматизации и интерфейса пользователя к ней
 \item Реализация системы автоматизации обработки данных
\end{itemize}


Исследование в работе алгоритмов последующей обработки экспериментальных данных поглощения мюонов
снижает погрешность при измерениях, упрощает работу оператора системы, кроме того 
сокращает время измерения плотности грунта.


Данная работа описана в трех главах. В первой главе описывается анализ предметной области, построение физической 
модели рождения атмосферных мюонов и их переноса в веществе, формирование требований к системе автоматизации. Во второй главе
описываются алгоритмы генерации и обработки данных, архитектура системы автоматизации. В третьей главе описаны вопросы реализации, 
результаты тестирования на реальных данных.

\clearpage