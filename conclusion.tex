\chapter*{Заключение}						% Заголовок
\addcontentsline{toc}{chapter}{Заключение}	% Добавляем его в оглавление

Была разработана система автоматизации обработки данных переносного мюонного плотномера, позволяющая автоматически строить тарировочные кривые с
погрешностью аппроксимации на уровне 2\%. Система позволяет определять по тарировочным данным и измерению потока мюонов 
плотность породы в объеме. 

Разработан комбинированный алгоритм, аппроксимирующий тарировочные кривые и минимизирующий оценку по методу МНГНК. Оценен эффективный объем, в котором проводится измерение плотности грунта. Был разработан 
метод измерения плотности, использующий данные серии измерений с целью уменьшения погрешности и локализации 
неоднородностей.

На данный момент разработан пилотный вариант мюонного плотномера, в дальнейшем планируется: добавить автоматическое 
управление прибором и процессом измерения, реализовать метод поиска неоднородностей.

\section*{Список публикаций по теме работы}						% Заголовок
\addcontentsline{toc}{chapter}{Список публикаций по теме работы}

\begin{enumerate}

\item Зюбин В. Е. , Сизов М. М. : «Алгоритм аппроксимации тарировочных данных переносного плотномера» XVII Международная открытая научная конференция «Современные проблемы информатизации-2013»

\item Сизов М. М. : «Разработка алгоритма построения тарировочной кривой для переносного мюонного плотномера» 51 Международная Научная Студенческая Конференция 
\end{enumerate}

\clearpage